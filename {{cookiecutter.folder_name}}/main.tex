% 
% ======================================================================
\RequirePackage{docswitch}
\setjournal{\flag}

\documentclass[\docopts]{\docclass}

% You could also define the document class directly
%\documentclass[]{emulateapj}

\input{macros}

\usepackage{graphicx}
\graphicspath{{./}{./figures/}}
\bibliographystyle{apj}

% ======================================================================

\begin{document}

\title{ {{ cookiecutter.title }} }

\maketitlepre

\begin{abstract}

{{ cookiecutter.description }}

\end{abstract}

% Keywords are ignored in the LSST DESC Note style:
\dockeys{latex: templates, papers: awesome}

\maketitlepost

% ----------------------------------------------------------------------

\section{Introduction}
\label{sec:intro}

This is a paper and note template for the LSST DESC \citep{Overview,ScienceBook,WhitePaper}.
You can delete all this tutorial text whenever you like.

Eventually it will be possible to switch between various \LaTeX\xspace styles for internal notes and peer reviewed journals templates.
The base switch is between \code{aastex.cls} and \code{revtex.cls}; however, facilities are also provided for \code{emulateapj.cls} and \code{mnras.cls}.\footnote{The \code{mnras.cls} class file is a bit odd...}
Documents can be compiled using the provided \code{Makefile} with several options: \code{make apj}, \code{make apjl}, \code{make prd}, and \code{make mnras}.
There are some oddities when changing between templates, so please be patient while we try to work these out.


% ----------------------------------------------------------------------

\section{Commands}
\label{sec:commands}

There are a number of useful \LaTeX\xspace commands predefined in \code{macros.tex}.
Notice that the section labels are prefixed with \code{sec:} to allow the use of the \verb=\secref= command to reference a section (\ie, \secref{intro}).
Figures can be referenced with the \verb=\figref= command, which assumes that the figure label is prefixed with \code{fig:}.
In \figref{example} we show an example figure.
You'll notice that the actual figure file is found in the \code{figures} directory.
However, because we have specified this directory in our \verb=\graphicspath= we do not need to explicitly specify the path to the image.

The \code{macros.tex} package also contains some conventional scientific units like \angstrom, \GeV, \Msun, etc. and some editorial tools for highlighting \FIXME{issues}, \CHECK{text to be checked}, \COMMENT{comments}, and \NEW{new additions}.


% ----------------------------------------------------------------------

\section{Methods}
\label{sec:methods}

Similar to the figure before, here we have included a table of data from \code{tables/table.tex}.
Notice that again we are able to reference \tabref{example} with the \verb=\tabref= command using the \code{tab:} prefix.
Also notice that we haven't needed to specify the full path to the table because in the \code{Makefile} we include \code{./tables} directory in the \code{\$TEXINPUTS} environment variable.

\input{table}

Equations appear as follows, and can be referred to as, for example, \eqnref{example} -- just as for tables, we use the \verb=\eqnref= command using the \code{eqn:} prefix.
\begin{equation}
  \label{eqn:example}
  \langle f(k) \rangle = \frac{ \sum_{t=0}^{N}f(t,k) }{N}
\end{equation}


% ----------------------------------------------------------------------

\section{Results}
\label{sec:results}

\figref{example} shows an example figure, referred to with the \verb=\figref= command and the \code{fig:} prefix.

\begin{figure}
\includegraphics[width=0.9\columnwidth]{example.jpg}
\caption{An example figure. \label{fig:example}}
\end{figure}


% ----------------------------------------------------------------------

\section{Discussion}
\label{sec:discussion}

If you are planning on committing your paper to GitHub, it's a good idea to write your tex as one sentence per line.
This allows for an easier \code{diff} of changes.
It also makes sense to think of latex as \emph{code}, and sentences as logical statements, occupying one line each.
Each line must ``compile'' in the mind of the reader.


% ----------------------------------------------------------------------

\section{Conclusions}
\label{sec:conclusions}

Here's a summary of what we just reported.

We can draw the following well-organized and neatly-formatted conclusions:
\begin{itemize}
  \item This is important.
  \item We can measure some number with some precision.
  \item This has some implications.
\end{itemize}

Here are some parting thoughts.


% ----------------------------------------------------------------------

\subsection*{Acknowledgments}

Here is where you should add your specific acknowledgments, remembering that some standard thanks will be added via the \code{acknowledgments.tex} file.

% 
We acknowledge many great people.
% 


%{\it Facilities:} \facility{LSST}

\bibliography{main}

\end{document}
% ======================================================================
% 
